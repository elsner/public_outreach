%
% Copyright (C) 2018 Franz Elsner <f.elsner[AT]mpa-garching.mpg.de>
%

\documentclass[12pt,twoside,a4paper]{article}

\usepackage[ngerman]{babel}
\usepackage{graphicx}
\usepackage{amsmath}
\usepackage{amssymb}
\usepackage{fontawesome}
\usepackage{pstricks}
\usepackage{url}
\usepackage{listings}
\usepackage{xspace}
\xspaceaddexceptions{]\}}

\lstset{frame=tb,
  language=Python,
  basicstyle=\ttm,
  aboveskip=1ex,
  belowskip=1ex,
  showstringspaces=false,
  columns=flexible,
  basicstyle={\small\ttfamily},
  numbers=none,
  breaklines=false,
  breakatwhitespace=true,
  tabsize=3
}

\newcommand{\cmb}{\emph{CMB}\xspace}
\newcommand{\plus}{\raisebox{-0.3ex}{\faPlusSquareO}\xspace}


\begin{document}

\title{Projekt zur kosmischen Mikrowellenhintergrundstrahlung}

\author{Franz Elsner\\felsner[AT]mpa-garching.mpg.de}

\maketitle

\begin{center}
  \includegraphics[width=\textwidth]{planck}
\end{center}



\section{\"Ubersicht}

Die kosmische Mikrowellenhintergrundstrahlung (englisch: \emph{cosmic
  microwave background radiation}, oder abgek\"urzt \cmb) ist ein
Relikt des Urknalls. Sie gibt uns direkte Hinweise auf die
Entstehungsgeschichte unseres Universums. In diesem Projekt versuchen
wir, einige davon zu reproduzieren.


\subsection{Aufgaben}

\begin{enumerate}
\item Recherchiere im Internet und versuche herauszufinden, wie alt
  das Universum war, als das Licht des \cmb emittiert wurde. Welches
  Alter hat das Universum heute und wie alt ist im Vergleich dazu
  beispielsweise unsere Erde?
 \item Astronomen haben eine eigene Einheit eingef\"uhrt, um die
   typischerweise sehr gro{\ss}en Distanzen im Universum zu messen: das
   \emph{Parsec}. Das Licht des Mikrowellenhintergrunds hat auf seinem
   Weg zu uns eine Distanz von etwa 14 Gigaparsec zur\"uckgelegt, also
   14 Milliarden Parsec. Wieviele Kilometer sind das?
\end{enumerate}



\section{Rotverschiebung}

Die kosmische Hintegrundstrahlung wurde im fr\"uhen Universum als
sogenannte Schwarz\-k\"orperstrahlung mit einer Temperatur von etwa
3000 Kelvin emittiert (das sind n\"aherungsweise $2700 \,
^{\circ}\mathrm{C}$). Das entspricht in etwa der Temperatur und Farbe
des Gl\"uhfadens einer herk\"omm\-lichen Gl\"uhlampe. Heute k\"onnen
wir die Hintergrundstrahlung jedoch nicht mehr mit dem blo{\ss}em Auge
erkennen, da das Licht einen Gro{\ss}teil seiner Energie
eingeb\"u{\ss}t hat.


\subsection{Aufgaben}

\begin{enumerate}
  \setcounter{enumi}{2}
\item Auch jetzt noch l\"asst sich der \cmb als
  Schwarzk\"orperstrahlung beschreiben. Welche Temperatur w\"urden wir
  ihm heute zuordnen?
\item Wurde die Strahlung also zu k\"urzeren (Richtung
  R\"ontgenstrahlen) oder l\"angeren Wellenl\"angen (in den
  Radiobereich) verschoben? Wieso sprechen wir von der
  \emph{Mikrowellen}hintergrundstrahlung?
\item Wissenschaftler glauben, dass das Universum in einem hei{\ss}en
  Urknall entstanden ist und sich seitdem ausdehnt, was unter anderem
  eine generelle Abk\"uhlung zur Folge hat. Was glaubst Du -- steht
  die Mikrowellenhintergrundstrahlung mit dieser Annahme im
  Widerspruch?
\end{enumerate}



\section{Darstellung des Datensatzes}

Wir wollen uns nun die Mikrowellenhintergrundstrahlung genauer
ansehen. Dazu verwenden wir Messungen, die vor wenigen Jahren mit dem
Weltraumteleskop \emph{Planck} gemacht worden sind. F\"ur die Analyse
steht ein Softwareprogramm zur Verf\"ugung, das auf den Rechnern
installiert ist.


\subsection{Aufgaben}

\begin{enumerate}
  \setcounter{enumi}{5}
\item Starte das bereitgestellte Python Notebook durch einen
  Doppelklick auf das Desktop Icon (das wird einen Moment dauern) und
  dr\"ucke dann die Tastenkombination `Shift + Enter' um die
  vordefinierten Funktionen verf\"ugbar zu machen, die wir f\"ur
  unsere Analyse brauchen. Lade dann den \cmb Datensatz. Schreibe dazu
  den untenstehenden Befehl in eine leere Zelle am Ende des Notebooks
  (durch dr\"ucken des Knopfs mit dem Zeichen \plus kannst Du
  jederzeit eine neue Zelle anh\"angen). F\"uhre Deinen Befehl dann
  wieder durch die Tastenkombination `Shift + Enter' aus.

  Zeilen, die mit dem Zeichen "\texttt{\#}" beginnen sind Kommentare
  und werden vom Computer ignoriert, sie dienen lediglich der
  Dokumentation Deines Programms.
\end{enumerate}
\vspace{-1ex}
\begin{lstlisting}
  # Load CMB data
  cmb_map = load_cmb_data()
\end{lstlisting}
\begin{enumerate}
  \setcounter{enumi}{6}
\item Generiere ein Bild des \cmb mit der zugeh\"origen Funktion, die
  Du wieder mit `Shift + Enter' ausf\"uhrst. Es handelt sich hierbei
  um eine Projektion des vollst\"andigen Himmels in eine
  zweidimensionale Ebene, die sich dann gut auf dem Bildschirm
  darstellen l\"asst (\"ahnlich der Konstruktion einer Weltkarte, die
  eine vereinfachende Projektion der etwa kugelf\"ormigen
  Erdoberfl\"ache ist). Das Bild der Karte ist in der Einheit
  \emph{Kelvin} angegeben, also einer Temperatur. Genauer gesagt zeigt
  die Karte Temperaturschwankungen um einen Mittelwert, der hier auf
  Null gesetzt wurde. Klicke mit der Maus auf verschiedene Regionen in
  der Karte, um sie zu vergr\"o{\ss}ern -- wie gro{\ss} sind die
  Abweichungen vom Mittelwert typischerweise? Vergleiche diesen
  Zahlenwert mit Deinem Ergebnis aus Aufgabe (3). Sind diese
  Schwankungen in jeder Himmelsrichtung \"ahnlich gro{\ss}?
\end{enumerate}
\vspace{-1ex}
\begin{lstlisting}
  # Plot CMB map
  plot_cmb_map(cmb_map, figure=1, title='CMB')
\end{lstlisting}




\section{Kosmologische Parameter}

Durch die Analyse der statistischen Eigenschaften der
Hintergrundstrahlung lassen sich genaue R\"uckschl\"usse auf
die grundlegende Beschaffenheit unseres Universums ziehen.


\subsection{Aufgaben}

\begin{enumerate}
  \setcounter{enumi}{7}
\item Wir k\"onnen die Hintergrundstrahlung benutzen, um Parameter zu
  bestimmen, die unser Universum beschreiben. Versuche ein Beispiel
  f\"ur einen solchen Parameter zu finden. Was beschreibt etwa die
  sogenannte \emph{Hubble Konstante}? Was glaubst Du -- hat sie auch
  einen Einfluss auf die Hintergrundstrahlung?
\item Das Programm erlaubt, den Effekt der kosmologischen Parameter
  auf die Hintergrundstrahlung zu simulieren. Verwende eine neue Zelle
  (\plus) und variiere mit den untenstehenden Befehlen einen der
  kosmologischen Parameter in dem angegebenen Zahlenbereich. Versuche
  zu beschreiben, wie sich das Bild der Hintergrundstrahlung dadurch
  ver\"andert.
\end{enumerate}
\vspace{-1ex}
\begin{lstlisting}
  # Try values for the spectral index between 0 and 2
  cosmology = {"scalar_spectral_index(1)": 0.1}

  # Rescale map to new cosmology
  rescaled_cmb_map = camb.rescale_cmb_map(cmb_map, cosmology)

  # Plot rescaled map
  plot_cmb_map(rescaled_cmb_map, figure=2, title='Rescaled CMB')
\end{lstlisting}
\vspace{-1ex}
\begin{enumerate}
\item[] Kannst Du einen Zahlenwert finden, f\"ur den das Bild der
  tats\"achlichen Karte aus Aufgabe (7) \"ahnelt? Damit hast Du
  erfolgreich einen kosmologischen Parameter unseres Universums
  bestimmt!
\setcounter{enumi}{9}
\item Von einer anderen Position im Weltall aus betrachtet
  (z.B.\ einer sehr weit entfernten Galaxie), ver\"andert sich das
  Aussehen des Mikrowellenhintergrunds, auch wenn sich die
  grundlegenden Eigenschaften des Universums nat\"urlich nicht
  ge\"andert haben. Erstelle eine neue Zelle (\plus) und nutze dann
  die Software, um eine alternative Realisation des \cmb auf dem
  Computer zu simulieren und darzustellen. Auch wenn sich das Bild nun
  grundlegend ver\"andert hat -- kannst Du vielleicht trotzdem
  \"Ubereinstimm\-ungen mit dem tats\"achlichen Datensatz finden?
\end{enumerate}
\vspace{-1ex}
\begin{lstlisting}
  # Simulate CMB map
  simulated_cmb_map = camb.simulate_cmb_map()

  # Plot simulated map
  plot_cmb_map(simulated_cmb_map, figure=3, title='Simulated CMB')
\end{lstlisting}



\section{Zusatzaufgabe: Analyse der Fluktuationen im \cmb}

Zur Analyse der Hintergrundstrahlung verwenden Wissenschaftler das
sogenannte \emph{power spectrum} (zu deutsch: spektrale
Leistungsdichte). Diese statistische Gr\"o{\ss}e beschreibt, wie stark
der Datensatz typischerweise in zwei Punkten variiert, die eine
bestimmte Distanz voneinander haben.

\begin{enumerate}
  \setcounter{enumi}{10}
\item Nutze die Software, um das \emph{power spectrum} des \cmb zu
  berechnen und darzustellen. Verwende daf\"ur die untenstehende
  Funktion in einer neuen Zelle (\plus), wobei Du die Variable `KARTE'
  mit der Karte ersetzen musst, die Du analysieren
  m\"ochtest. Beschreibe die Form der Kurve, die angezeigt wird.
\end{enumerate}
\vspace{-1ex}
\begin{lstlisting}
  plot_power_spectrum(KARTE, figure=4)
\end{lstlisting}
\begin{enumerate}
  \setcounter{enumi}{11}
\item Welchen Einfluss haben die kosmologischen Parameter auf das
  \emph{power spectrum}? Nutze die Software, um das \emph{power
    spectrum} einer modifizierten Karte des \cmb zu bestimmen, die Du
  in Aufgabe (9) berechnet hast. Wie hat es sich ver\"andert?
\item Vergleiche das \emph{power spectrum} des echten \cmb mit dem
  Deiner simulierten \cmb Karte aus Aufgabe (10). Was stellst Du hier
  fest?
\end{enumerate}

\end{document}
